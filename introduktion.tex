\chapter{Introduktion}
I efteråret 2024 blev der i Fysisk Auditorium givet en simpel mekanikopgave
omhandlende en masse, en fjder og en overflade. Opgaven var at finde ud af hvor
langt en kasse ville bevæge sig når den glider på en overflade og er påvirket af en fjder, 
som er monteret mellem siden af kassen og en væg. En simpel opgave der nemt kunne løses 
ved brug af energibevarelse, viste sig dog at være en større udfordring. Kassen vil stoppe 
når fjderkraften bliver lavere end gnidningsmodstanden. Dette sker ikke kun på et sted men
i et interval omkring hvilepositionen og vil derfor give betydeligt forskellige svar 
afhængningt af startpositionen. Dette er hvad der undersøges nermere her.
