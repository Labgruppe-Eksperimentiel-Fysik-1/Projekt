\chapter{Introduktion}
I efteråret 2024 blev der i Fysisk Auditorium givet en simpel mekanikopgave
omhandlende en masse, en fjeder og en overflade. Opgaven lød  på at bestemme hvor
langt en kasse ville bevæge sig når den glider på en overflade og er påvirket af en fjeder, 
som er monteret mellem siden af kassen og en væg. Tiltænkt som en simpel opgave der nemt kunne løses 
ved brug af energibevarelse, viste den sig dog at være en større udfordring, som selv opgavestilleren fik forkert til at starte med.
Kassen vil nemlig ikke stoppe i fjederens hvileposition, men i stedet når fjederkraften bliver lavere end den statiske gnidningsmodstand. Dette sker ikke kun på et sted men
i et interval omkring hvilepositionen og vil derfor give betydeligt forskellige svar 
afhængningt af startpositionen. Det var muligt for nogle studerende at finde frem til en løsning, 
og det er gyldigheden af denne formodede løsning, 
der undersøges nærmere i dette projekt.