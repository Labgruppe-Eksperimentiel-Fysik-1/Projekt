\chapter{Resultater}
Vi udførte 2 forsøg med 3 gentagelser af hver. Det første var at vi optrækkede loddet til dens fuldt udspændte position og så slap den. 
Her varierede vi normalkraften hvormed underlaget trykkede på loddet. Vi gjorde det i 4 omgange. Først med en normalkraft på 0, altså intet underlag, så med en normalkraft under massen af loddet, en cirka ligmed massen af loddet, og en der var over.
Herefter gennemførte vi det samme forsøg, men ved cirka den halve startamplitude.

Første skrift i behandlingen af optagelserne, var at køre dem igennem Tracker. 
Tracker er et program der kan bruges til at tracke objekter i en video, 
og vi brugte det derfor til at tracke bevægelsen af det røde bånd på loddet under svingningen.
Nulpunktet til trackningen, blev fundet ved at tracke de 2 ender af newtonmetrerne og så finde midtpunktet imellem dem, eftersom at de er lige stærke, så burde det være ligevægtspositionen.

