\chapter{Konklusion}
Det er i dette projekt blevet klart hvordan at den via gnindning dæmpede svingning er forskellig fra andre bekendte dæmpede oscillationer.
Den teoretiske analyse af problemet fandt frem til at massen ville oscillere med konstant periode, men at amplituden ville falde med en konstant værdi $\frac{2f_k}{k}$, og at bevægelse ikke altid ville standse i ligevægtspunktet for fjedrerne, 
men at den kunne stoppe indenfor et interval omkring ligevægtspunktet, hvor størrelsen af intervallet afhænger af styrken af den statiske gnindningskraft, og stedet den standser indenfor intervallet er bestemt af oscillationens startbetingelser.
Eksperimentielt har vi forsøgt at eftervise denne sammenhæng, men resultaterne stemte meget dårligt overens med forventningerne. Afvigelserne kunne dog forklares ved et malplaceret nulpunkt, der efter korrektion vidste det forventede. 
Hvis forsøget skulle gentages, ville det være vigtigt at sikre sig at man har bestemt ligevægtspositionen/nulpunktet korrekt, da stadig afvigende målinger herefter, ville betyde der var noget galt med hypotesen. 
Desuden ville det også være klogt at måle gnidningskoefficienterne for opstillingen, 
da disse kan bruges til yderligere at teste flere af forventningerne, 
som for eksempel om slutpositionen og dæmpningsfaktorens værdi svarer til det som teorien forudsiger. 