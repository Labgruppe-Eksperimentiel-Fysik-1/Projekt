\maketitle
%
\clearpage
\chapter{Abstract}
This rapport is written as a final project for the course Experimental Physics 1. 
It serves to analyse a seemingly simple mechanics problem given as an assignment on the first semester, regarding a mass connected to a spring, moving across a surface with fiction.
The analysis of the problem concludes that theoretically, the amplitude of the motion should dampen with a constant factor each rebound, and the motion should stop within an interval around the equilibrium position of the spring, 
where the exact position is determed by the initial conditions of the motion.
Experimentally, we conclude that the analysis performed in the paper might be correct, since the damping is constant, but with different values in the direction of motion.
Therefore we are able to achieve the expected result after an artifical offset, since we expect that out equilibrium position was not where it was supposed to be. 
This together with other uncertainties, and the fact that we did not measure other important variables that would be useful in testing the validity of our hypothesis, 
means that further experiments would be neccesarry to actually prove our theory correct.